%2multibyte Version: 5.50.0.2953 CodePage: 65001
% Required for inserting images

\documentclass{article}%
\usepackage{algorithm}
\usepackage{algpseudocode}
\usepackage{alphabeta}
\usepackage{amsmath}
\usepackage{amsfonts}
\usepackage{amssymb}
\usepackage{boxedminipage}
\usepackage[open,openlevel=1]{bookmark}
\usepackage{caption}
\usepackage{color}
\usepackage{graphicx}
\usepackage{listings}
\usepackage[most]{tcolorbox}
\usepackage{multirow}
\usepackage{tikz}%
\setcounter{MaxMatrixCols}{30}
%TCIDATA{OutputFilter=latex2.dll}
%TCIDATA{Version=5.50.0.2953}
%TCIDATA{Codepage=65001}
%TCIDATA{LastRevised=Thursday, November 14, 2024 11:40:41}
%TCIDATA{<META NAME="GraphicsSave" CONTENT="32">}
%TCIDATA{<META NAME="SaveForMode" CONTENT="1">}
%TCIDATA{BibliographyScheme=Manual}
%BeginMSIPreambleData
\providecommand{\U}[1]{\protect\rule{.1in}{.1in}}
%EndMSIPreambleData
\providecommand{\U}[1]{\protect \rule{.1in}{.1in}}
\numberwithin{equation}{section}
\newtheorem{theorem}{Theorem}[section]
\newtheorem{acknowledgement}[theorem]{Acknowledgement}
\newtheorem{ALGORITHM}[theorem]{Algorithm}
\newtheorem{axiom}[theorem]{Axiom}
\newtheorem{assumption}[theorem]{Assumption}
\newtheorem{case}[theorem]{Case}
\newtheorem{claim}[theorem]{Claim}
\newtheorem{conclusion}[theorem]{Conclusion}
\newtheorem{condition}[theorem]{Condition}
\newtheorem{conjecture}[theorem]{Conjecture}
\newtheorem{corollary}[theorem]{Corollary}
\newtheorem{criterion}[theorem]{Criterion}
\newtheorem{definition}[theorem]{Definition}
\newtheorem{example}[theorem]{Example}
\newtheorem{exercise}[theorem]{Exercise}
\newtheorem{lemma}[theorem]{Lemma}
\newtheorem{notation}[theorem]{Notation}
\newtheorem{problem}[theorem]{Problem}
\newtheorem{proposition}[theorem]{Proposition}
\newtheorem{question}[theorem]{Question}
\newtheorem{remark}[theorem]{Remark}
\newtheorem{solution}[theorem]{Solution}
\newtheorem{summary}[theorem]{Summary}
\newenvironment{proof}[1][Proof]{\noindent \textbf{#1.} }{\  \rule{0.5em}{0.5em}}
\setcounter{tocdepth}{6}
\setlength{\textwidth}{6.75in}
\setlength{\textheight}{9.00in}
\setlength{\oddsidemargin}{-.125 in}
\setlength{\evensidemargin}{-.125 in}
\setlength{\topmargin}{-.60 in}
\begin{document}

\title{An Evolutionary Duel Tournament}
\author{Ath. Kehagias }
\date{}
\maketitle

\section{Intro}

We study an evolutionary duel tournament which consists of the following components

\begin{enumerate}
\item The \emph{basic duel game} or \emph{match}, which is played between two
players. We study two variants:\ the \emph{discounted} and \emph{undiscounted} ones.

\item The \emph{meeting}, which consists of all possible matches between $N$ players.

\item The evolutionary \emph{tournament}, which consists of one meeting per
\emph{generation} of players. The outcomes of each meeting are used to update
the strategies used by the players in the meeting of next generation.
\end{enumerate}

Our main goal is the following:\ there exist $M$ strategies and each of $N$
players will use one of these strategies. The representation of strategies in
the next generation depends on their performance in the current generation.
More precisely, the performance of a strategy is proprtional to the total
payoff accumulated by all the players who have used this strategy. It is hoped
that, as time progresses, better performing strategies will be more heavily
represented and will eventually \textquotedblleft
exterminate\textquotedblright\ strategies with worse performance.

\section{The Discounted Basic Game}

\subsection{Rules of the Game\noindent}

The discounted base game is played between players $P_{1}$ and $P_{2}$; player
$P_{n}$ has \emph{kill probability }$p_{n}$. In each round the two players
shoot simultaneously, with $P_{n}$'s shooting probability $x_{n}$ being
determined by the \emph{strategy} he has adopted (a player can also choose to
\emph{abstain}, i.e., to not shoot). The game state is $\mathbf{s}=s_{1}s_{2}%
$, where $s_{n}=0$ (resp. $s_{n}=1$) means $P_{n}$ is dead (resp. alive). The
state set is
\[
S=\left\{  00,01,10,11\right\}
\]
and each state $\mathbf{s}$ yields \emph{round payoff} $q_{n}\left(
\mathbf{s}\right)  $ to $P_{n}$. We have
\begin{align*}
q_{1}\left(  00\right)   &  =0,\quad q_{1}\left(  01\right)  =0,\quad
q_{1}\left(  10\right)  =1,\quad q_{1}\left(  11\right)  =1,\\
q_{2}\left(  00\right)   &  =0,\quad q_{2}\left(  01\right)  =1,\quad
q_{2}\left(  10\right)  =0,\quad q_{1}\left(  11\right)  =1.
\end{align*}
The idea is that each player receives one payoff unit for each turn in which
he is alive. The \emph{total payoff} is%
\[
Q_{n}\left(  \mathbf{s}\left(  0\right)  \mathbf{s}\left(  1\right)
\mathbf{s}\left(  2\right)  ...\right)  =\sum_{t=0}^{\infty}\gamma^{t}%
q_{n}\left(  \mathbf{s}\left(  t\right)  \right)
\]
(i.e., it round payoff is discounted by a factor $\gamma\in\left(  0,1\right)
$). The \emph{expected total payoff} is%
\[
\overline{Q}_{n}=\mathbb{E}\left(  Q_{n}\left(  \mathbf{s}\left(  0\right)
\mathbf{s}\left(  1\right)  \mathbf{s}\left(  2\right)  ...\right)  \right)
\]
where the expectation is taken with respect to the probability determined by
the players' strategies.

\subsection{Payoff Equation under Stationary Strategies}

Suppose that $P_{n}$ uses a stationary strategy $\sigma\left(  \mathbf{s}%
\right)  =\left(  x_{n1},x_{n2}\right)  $, where
\[
x_{nm}=\Pr\left(  P_{n}\text{ shoots at }P_{m}\right)
\]
with \textquotedblleft$P_{n}$ shoots at $P_{n}$\textquotedblright\ being
understood as \textquotedblleft$P_{n}$ abstains\textquotedblright. Since we
must have $x_{n1}+x_{n2}=1$, we can parametrize $P_{n}$'s stationary strategy
by a single number $x_{n}$, the probability of shooting at the other player.
We define%
\[
V_{\mathbf{s}}^{n}\left(  x_{1},x_{2}\right)  =\text{\textquotedblleft}%
P_{n}\text{'s expected payoff when starting at state }\mathbf{s}\text{ and
shoooting probs are }x_{1},x_{2}\text{\textquotedblright.}%
\]
We have $\sum_{t=0}^{\infty}\gamma^{t}=\allowbreak-\frac{1}{\gamma-1}$
\[
V_{00}^{1}=0,\quad V_{01}^{1}=0,\quad V_{10}^{1}=\sum_{t=0}^{\infty}\gamma
^{t}\cdot1=\frac{1}{1-\gamma}.
\]
For $V_{11}^{1}$ we have the following equation%

\begin{align*}
V_{11}^{1}  &  =x_{1}p_{1}x_{2}p_{2}V_{00}^{1}+\left(  x_{1}\left(
1-p_{1}\right)  +\left(  1-x_{1}\right)  \right)  x_{2}p_{2}V_{01}^{1}%
+x_{1}p_{1}\left(  x_{2}\left(  1-p_{2}\right)  +\left(  1-x_{2}\right)
\right)  V_{10}^{1}\\
&  +\left(  x_{1}\left(  1-p_{1}\right)  +\left(  1-x_{1}\right)  \right)
\left(  x_{2}\left(  1-p_{2}\right)  +\left(  1-x_{2}\right)  \right)  \left(
1+\gamma V_{11}^{1}\right)
\end{align*}
which becomes%

\[
V_{11}^{1}=x_{1}p_{1}\left(  x_{2}\left(  1-p_{2}\right)  +\left(
1-x_{2}\right)  \right)  \frac{1}{1-\gamma}+\left(  x_{1}\left(
1-p_{1}\right)  +\left(  1-x_{1}\right)  \right)  \left(  x_{2}\left(
1-p_{2}\right)  +\left(  1-x_{2}\right)  \right)  \left(  1+\gamma V_{11}%
^{1}\right)
\]
Solving, when $\max\left(  x_{1},x_{2}\right)  >0$ we get%
\[
V_{11}^{1}\left(  x_{1},x_{2}\right)  =\frac{\left(  1-x_{2}p_{2}\right)
\left(  1-\gamma\left(  1-x_{1}p_{1}\right)  \right)  }{\left(  1-\gamma
\right)  \left(  1-\gamma\left(  1-x_{1}p_{1}\right)  \left(  1-x_{2}%
p_{2}\right)  \right)  }%
\]
and when $\max\left(  x_{1},x_{2}\right)  =0$
\[
V_{11}^{1}\left(  x_{1},x_{2}\right)  =\frac{1}{1-\gamma}.
\]
Repeating the process for $V_{11}^{2}$ we get
\begin{align*}
V_{00}^{2}  &  =0,\quad V_{01}^{2}=0,\quad V_{10}^{2}=\frac{1}{1-\gamma}\\
V_{11}^{2}\left(  x_{1},x_{2}\right)   &  =\frac{\left(  1-x_{1}p_{1}\right)
\left(  1-\gamma\left(  1-x_{2}p_{2}\right)  \right)  }{\left(  1-\gamma
\right)  \left(  1-\gamma\left(  1-x_{1}p_{1}\right)  \left(  1-x_{2}%
p_{2}\right)  \right)  }%
\end{align*}


\subsection{Stationary Equilibria}

We will now show that the game has two Nash equilibria in stationary strategies.

\begin{enumerate}
\item When $\max\left(  x_{1},x_{2}\right)  >1$:%
\begin{align*}
\frac{dV_{11}^{1}}{dx_{1}}  &  =D_{x_{1}}\left(  \frac{\left(  1-x_{2}%
p_{2}\right)  \left(  1-\gamma\left(  1-x_{1}p_{1}\right)  \right)  }{\left(
1-\gamma\right)  \left(  1-\gamma\left(  1-x_{1}p_{1}\right)  \left(
1-x_{2}p_{2}\right)  \right)  }\right)  =\allowbreak\frac{\left(  1-x_{2}%
p_{2}\right)  \gamma p_{1}p_{2}x_{2}}{\left(  1-\gamma\right)  \left(
1-\gamma\left(  1-x_{2}p_{2}\right)  \left(  1-x_{1}p_{1}\right)  \right)
^{2}}\allowbreak>0\\
\frac{dV_{11}^{2}}{dx_{2}}  &  =D_{x_{2}}\left(  \frac{\left(  1-x_{1}%
p_{1}\right)  \left(  1-\gamma\left(  1-x_{2}p_{2}\right)  \right)  }{\left(
1-\gamma\right)  \left(  1-\gamma\left(  1-x_{1}p_{1}\right)  \left(
1-x_{2}p_{2}\right)  \right)  }\right)  =\allowbreak\frac{\left(  1-x_{1}%
p_{1}\right)  \gamma p_{1}p_{2}x_{1}}{\left(  1-\gamma\right)  \left(
1-\gamma\left(  1-x_{2}p_{2}\right)  \left(  1-x_{1}p_{1}\right)  \right)
^{2}}\allowbreak>0
\end{align*}
Hence, whenever $x_{-n}>0$, $P_{n}$ has motive to increase $x_{n}$ as much as
possible. Consequently $\left(  \widehat{x}_{1},\widehat{x}_{2}\right)
=\left(  1,1\right)  $ is a NE.

\item When $\left(  \widehat{x}_{1},\widehat{x}_{2}\right)  =\left(
0,0\right)  $, we have $\frac{\left(  1-\gamma\right)  }{\left(
1-\gamma\right)  \left(  1-\gamma\right)  }-\frac{\left(  1-\gamma\left(
1-x_{1}p_{1}\right)  \right)  }{\left(  1-\gamma\right)  \left(
1-\gamma\left(  1-x_{1}p_{1}\right)  \right)  }$%
\begin{align*}
V_{11}^{1}\left(  0,0\right)  -V_{11}^{1}\left(  x_{1},0\right)   &
=\frac{\left(  1-\gamma\right)  }{\left(  1-\gamma\right)  \left(
1-\gamma\right)  }-\frac{\left(  1-\gamma\left(  1-x_{1}p_{1}\right)  \right)
}{\left(  1-\gamma\right)  \left(  1-\gamma\left(  1-x_{1}p_{1}\right)
\right)  }=0\\
V_{11}^{2}\left(  0,0\right)  -V_{11}^{2}\left(  0,x_{2}\right)   &  =0
\end{align*}
Hence, whenever $P_{n}$ has no motive to move away from $x_{n}=0$.
Consequently $\left(  \widehat{x}_{1},\widehat{x}_{2}\right)  =\left(
0,0\right)  $ is a NE.
\end{enumerate}

\subsection{Nonstationary Equilibria}

Let $\sigma_{G}$ be the no-shoot strategy with grim retaliation. We will show
that $\left(  \sigma_{G},\sigma_{G}\right)  $ is a NE. Suppose both players
are using $\sigma_{G}$; then $P_{1}$ gets the payoff
\[
\widehat{V}_{11}^{1}=V_{11}^{1}\left(  0,0\right)  =\frac{1}{1-\gamma}.
\]
Now $P_{1}$ considers deviating. If he shoots once at some round, then $P_{2}$
will keep shooting back in all subsequent rounds and then $P_{1}$'s best
strategy is to also keep shooting. He only needs to consider the deviating
strategy in which he shoots at the first round, with probability one. This
deviation yields:
\begin{align*}
\widetilde{V}_{11}^{1}  &  =p_{1}V_{11}^{1}\left(  1,0\right)  +\left(
1-p_{1}\right)  V_{11}^{1}\left(  1,1\right) \\
&  =\frac{1-\gamma\left(  1-p_{1}\right)  \left(  1-p_{2}\right)
-p_{2}\left(  1-p_{1}\right)  }{\left(  1-\gamma\right)  \left(
1-\gamma\left(  1-p_{1}\right)  \left(  1-p_{2}\right)  \right)  }%
\end{align*}
Then we have%
\begin{align*}
\widehat{V}_{11}^{1}-\widetilde{V}_{11}^{1}  &  =\frac{1}{1-\gamma}%
-\frac{1-\gamma\left(  1-p_{1}\right)  \left(  1-p_{2}\right)  -p_{2}\left(
1-p_{1}\right)  }{\left(  1-\gamma\right)  \left(  1-\gamma\left(
1-p_{1}\right)  \left(  1-p_{2}\right)  \right)  }\\
&  =\allowbreak\frac{p_{2}\left(  1-p_{1}\right)  }{\left(  1-\gamma\right)
\left(  1-\gamma\left(  1-p_{1}\right)  \left(  1-p_{2}\right)  \right)  }>0
\end{align*}
Hence $\left(  \sigma_{G},\sigma_{G}\right)  $ is a NE for all $\gamma
,p_{1},p_{2}$ values.

\section{The Undiscounted Basic Game}

\subsection{Rules of the Game\noindent}

The undiscounted game is played just like the discounted one, but the payoffs
are different. We have a \emph{kill payoff} $a>0$ and a \emph{death cost}
$-b<0$ and there is no discounting. The \emph{round payoff} $q_{n}\left(
\mathbf{s}\right)  $ to $P_{n}$ is%
\begin{align*}
q_{1}\left(  00\right)   &  =a-b,\quad q_{1}\left(  01\right)  =-b,\quad
q_{1}\left(  10\right)  =a,\quad q_{1}\left(  11\right)  =0,\\
q_{2}\left(  00\right)   &  =a-b,\quad q_{2}\left(  01\right)  =a,\quad
q_{2}\left(  10\right)  =-b,\quad q_{1}\left(  11\right)  =0.
\end{align*}
The \emph{total payoff} is%
\[
Q_{n}\left(  \mathbf{s}\left(  0\right)  \mathbf{s}\left(  1\right)
\mathbf{s}\left(  2\right)  ...\right)  =\sum_{t=0}^{\infty}q_{n}\left(
\mathbf{s}\left(  t\right)  \right)
\]
and the \emph{expected total payoff} is%
\[
\overline{Q}_{n}=\mathbb{E}\left(  Q_{n}\left(  \mathbf{s}\left(  0\right)
\mathbf{s}\left(  1\right)  \mathbf{s}\left(  2\right)  ...\right)  \right)
.
\]


\subsection{Payoff under Stationary Strategies}

With the same notations as previously we have%
\[
V_{00}^{1}=a-b,\qquad V_{01}^{1}=-b,\qquad V_{10}^{1}=a.
\]
The payoff equation for $V_{11}^{1}$ is:%

\begin{align*}
V_{11}^{1}  &  =x_{1}p_{1}x_{2}p_{2}V_{00}^{1}+\left(  x_{1}\left(
1-p_{1}\right)  +\left(  1-x_{1}\right)  \right)  x_{2}p_{2}V_{01}^{1}%
+x_{1}p_{1}\left(  x_{2}\left(  1-p_{2}\right)  +\left(  1-x_{2}\right)
\right)  V_{10}^{1}\\
&  +\left(  x_{1}\left(  1-p_{1}\right)  +\left(  1-x_{1}\right)  \right)
\left(  x_{2}\left(  1-p_{2}\right)  +\left(  1-x_{2}\right)  \right)
V_{11}^{1}%
\end{align*}
or%

\begin{align*}
V_{11}^{1}  &  =x_{1}p_{1}x_{2}p_{2}\left(  a-b\right)  -\left(  x_{1}\left(
1-p_{1}\right)  +\left(  1-x_{1}\right)  \right)  x_{2}p_{2}b+x_{1}%
p_{1}\left(  x_{2}\left(  1-p_{2}\right)  +\left(  1-x_{2}\right)  \right)
a\\
&  +\left(  x_{1}\left(  1-p_{1}\right)  +\left(  1-x_{1}\right)  \right)
\left(  x_{2}\left(  1-p_{2}\right)  +\left(  1-x_{2}\right)  \right)
V_{11}^{1}%
\end{align*}
Solution is:
\[
V_{11}^{1}=\frac{x_{1}p_{1}a-x_{2}p_{2}b}{x_{1}p_{1}+x_{2}p_{2}-x_{1}%
p_{1}x_{2}p_{2}}%
\]
Similarly, for $P_{2}$, we get
\[
V_{11}^{2}=\frac{x_{2}p_{2}a-x_{1}p_{1}b}{x_{1}p_{1}+x_{2}p_{2}-x_{1}%
p_{1}x_{2}p_{2}}%
\]


\subsection{Stationary Equilibria}

The stationary unique NE\ is $\left(  \widehat{x}_{1},\widehat{x}_{2}\right)
=\left(  1,1\right)  $ because of the following. When $\max\left(  x_{1}%
,x_{2}\right)  >1$:%

\begin{align*}
\frac{dV_{11}^{1}}{dx_{1}}  &  =D_{x_{1}}\left(  \frac{x_{2}p_{2}b-x_{1}%
p_{1}a}{x_{1}p_{1}x_{2}p_{2}-x_{1}p_{1}-x_{2}p_{2}}\right) \\
&  =\allowbreak-p_{1}x_{2}p_{2}\frac{-a+x_{2}p_{2}b-b}{\left(  x_{1}p_{1}%
x_{2}p_{2}-x_{1}p_{1}-x_{2}p_{2}\right)  ^{2}}=\frac{p_{1}x_{2}p_{2}\left(
a+b\left(  1-x_{2}p_{2}\right)  \right)  }{\left(  x_{1}p_{1}x_{2}p_{2}%
-x_{1}p_{1}-x_{2}p_{2}\right)  ^{2}}>0
\end{align*}
%

\begin{align*}
\frac{dV_{11}^{2}}{dx_{1}}  &  =D_{x_{2}}\left(  -\frac{x_{2}p_{2}a-x_{1}%
p_{1}b}{x_{1}p_{1}x_{2}p_{2}-x_{1}p_{1}-x_{2}p_{2}}\right) \\
&  =\allowbreak-p_{2}x_{1}p_{1}\frac{-a+x_{1}p_{1}b-b}{\left(  x_{1}p_{1}%
x_{2}p_{2}-x_{1}p_{1}-x_{2}p_{2}\right)  ^{2}}=\frac{p_{2}x_{1}p_{1}\left(
a+b\left(  1-x_{1}p_{1}\right)  \right)  }{\left(  x_{1}p_{1}x_{2}p_{2}%
-x_{1}p_{1}-x_{2}p_{2}\right)  ^{2}}>0
\end{align*}
So $P_{1}$ (resp. $P_{2}$)\ must take $x_{1}=1$ (resp. $x_{2}=1$). On the
other hand, wWhen $x_{1}=x_{2}=0$ and $P_{1}$ deviates to $x_{1}^{\prime}>0$
we get%
\[
V_{11}^{1}\left(  0,0\right)  -V_{11}^{1}\left(  x_{1},0\right)  =-a<0
\]
So $P_{1}$ (resp. $P_{2}$)\ has motive to deviate from $x_{1}=0$ (resp. from
$x_{2}=0$).

\subsection{Nonstationary Equilibria}

Let $\sigma_{G}$ be the no-shoot strategy with grim retaliation. We will show
that $\left(  \sigma_{G},\sigma_{G}\right)  $ is a NE. Suppose both players
are using $\sigma_{G}$; then $P_{1}$ gets the payoff
\[
\widehat{V}_{11}^{1}=V_{11}^{1}\left(  0,0\right)  =0.
\]
Now $P_{1}$ considers deviating. If he shoots once at some round, then $P_{2}$
will keep shooting back in all subsequent rounds and then $P_{1}$'s best
strategy is to also keep shooting. He only needs to consider the deviating
strategy in which he shoots at the first round, with probability one. This
deviation yields:
\begin{align*}
\widetilde{V}_{11}^{1}  &  =p_{1}V_{11}^{1}\left(  1,0\right)  +\left(
1-p_{1}\right)  V_{11}^{1}\left(  1,1\right) \\
&  =\frac{-ap_{1}^{2}p_{2}+\left(  (a+b)p_{2}+a\right)  p_{1}-bp_{2}}%
{(p_{2}-1)p_{1}-p_{2}}.
\end{align*}
Then we have%
\[
\widehat{V}_{11}^{1}-\widetilde{V}_{11}^{1}=\frac{-a\left(  p_{1}+p_{1}%
p_{2}\left(  1-p_{2}\right)  \right)  +bp_{2}\left(  1-p_{1}\right)
}{(1-p_{2})p_{1}+p_{2}}%
\]
and for a NE we must have
\[
bp_{2}\left(  1-p_{1}\right)  >a\left(  p_{1}+p_{1}p_{2}\left(  1-p_{2}%
\right)  \right)
\]
and, from the similar condition for $P_{2}$,%
\[
bp_{1}\left(  1-p_{2}\right)  >a\left(  p_{2}+p_{1}p_{2}\left(  1-p_{1}%
\right)  \right)
\]
Combining these we get that $\left(  \sigma_{G},\sigma_{G}\right)  $ is a
NE\ if%
\[
\frac{b}{a}>\max\left\{  \frac{p_{1}+p_{1}p_{2}\left(  1-p_{2}\right)  }%
{p_{2}\left(  1-p_{1}\right)  }\text{,}\frac{p_{2}+p_{1}p_{2}\left(
1-p_{1}\right)  }{p_{1}\left(  1-p_{2}\right)  }\right\}  .
\]


\section{The Evolutionary Tournament}

\subsection{Fitness Based Tournament}

This tournament consists of $J$ generations, each of which involves $N$
players and $M$ strategies. In each generation every player plays a duel
against every other player; the duel can be either discounted or undiscounted.
The idea is that each strategy is evaluated by its \emph{fitness}, i.e., its
performance (summed over all players who use the strategy)\ against all
players and the fitness determines the proportion of each strategy in the
nextgeneration. The tournament is described by the following pseudocode.

\begin{algorithm}[h!]
\caption{Fitness Based Evolutionary Tournament}\label{LrngAlg01}
\begin{algorithmic}[1]
\State \textbf{Input}: kill prob. $p$, discount factor $\gamma$, no. of players $N$,
strategy set $\mathcal{S}$, no. of generations $J$
\For{$n\in \{1,2,...,N\}$}
\State $P_n$ selects randomly $\sigma_n\in\mathcal{S}$
\EndFor
\For{$j\in \{1,2,...,J\}$}
\For{$(n_1,n_2)\in \{1,...,N\}\times\{1,...,N\}$}
\State $P_{n_1}$ and $P_{n_2}$ fight a duel
\EndFor
\For{$n\in \{1,...,N\}$}
\State $A^j_n$ is $P_n$'s total payoff from all duels he fought
\EndFor
\For{$\sigma\in \mathcal{S}$}
\State $B^j_\sigma=\sum_{n:P_n \text{ uses }\sigma}A^j_n$
\EndFor
\For{$\sigma\in \mathcal{S}$}
\State $F^j_\sigma=\frac{B^j_\sigma}{\sum_{\sigma\in \mathcal{S}}B^j_\sigma}$ is the fitness of $\sigma$
\EndFor
\For{$(n_1,n_2)\in \{1,...,N\}\times\{1,...,N\}$}
\State $P_n$ selects $\sigma\in\mathcal{S}$ with prob. $F^j_\sigma$
\EndFor
\EndFor
\State \Return $\left(F^J_\sigma\right)_{\sigma\in \mathcal{S}}$
\end{algorithmic}
\end{algorithm}


\bigskip

\noindent It is also possible to modify the algorithm so that strategies
propagate into the next generation acording to individual player performance.
Initial experiments indicate that this tournament gives reasonable results
when applied to the discounted game, but performs terribly when applied to the
undiscounted one.

\subsection{Imitation Based Tournament}

This tournament also involves $N$ players, $M$ strategies and $J$ generations,
and in each generation every player plays a duel against every other player;
the duel can be either discounted or undiscounted (so far I\ have only tested
this with the undiscounted game). The idea here is that, in every generation
each player has a probability $\pi$ of switching his strategy to that of the
best performing player. This is a simple evolutiomnary dynamic and it should
be easy to establish its properties (and also for general tournaments, not
just for duels). The tournament is described by the following pseudocode.

\begin{algorithm}[h!]
\caption{Imitation Based Evolutionary Tournament}\label{LrngAlg01}
\begin{algorithmic}[1]
\State \textbf{Input}: kill prob. $p$, kill gain $a$, death loss $b$, no. of players $N$,
strategy set $\mathcal{S}$, mutation prob. $\pi$, no. of generations $J$
\For{$n\in \{1,2,...,N\}$}
\State $P_n$ selects randomly $\sigma_n\in\mathcal{S}$
\EndFor
\For{$j\in \{1,2,...,J\}$}
\For{$(n_1,n_2)\in \{1,...,N\}\times\{1,...,N\}$}
\State $P_{n_1}$ and $P_{n_2}$ fight a duel
\EndFor
\EndFor
\For{$n\in \{1,...,N\}$}
\State $A^j_n$ is $P_n$'s total payoff from all duels he fought
\EndFor
\State $\widehat{n}=\arg\max_n A^j_n$
\For{$n\in \{1,2,...,N\}$}
\State $P_n$ switches to $P_{\widehat{n}}$'s strategy w.p. $\pi$.
\EndFor
\State \Return The players' strategies
\end{algorithmic}
\end{algorithm}


\bigskip

\newpage

\section{Additional Remarks}

\begin{enumerate}
\item We must perform more experiments.

\begin{enumerate}
\item Apply imitation based tournament to the undiscounted game.

\item Use more strategies.

\item Compare performance when using theoretical payoff formulas vs.
simulation-obtained payoffs.

\item Try playing \emph{only one match} (two players) per generation. Or, more
generally, $K\leq N^{2}$ matches.

\item Compare performance of player evaluation vs. strategy evaluation. Why
don' these give identical results?
\end{enumerate}

\item Fitness functions:\ 

\begin{enumerate}
\item Fitness functions apprently cannot handle well \emph{negative scores}.
Can we improve this?
\end{enumerate}

\item Extensions to tournaments on \emph{graphs}.

\begin{enumerate}
\item When the graph is a clique, each player chooses the best overall strategy.

\item In graph, players should choose the best player / strategy in their neighborhood.

\item For general graph, we could compute a local fitness function, using
payoffs of each vertex neighborhood.
\end{enumerate}

\item Need to study more evolutionary game theory.

\begin{enumerate}
\item What are \ the properties of evolutionary stable equilibria?

\item What are alternative evolutionary dynamics?
\end{enumerate}

\item \emph{Mathematical} extensions.

\begin{enumerate}
\item Model the change of strategy proportions across generations (ODE? MC?).

\item In the fitness tournament we essentially have an update
\begin{equation}
z_{m,t+1}=\frac{F\left(  \sum_{k\neq m}z_{k,t}Q_{mk}\right)  }{\sum_{m=1}%
^{M}F\left(  \sum_{k\neq m}z_{k,t}Q_{mk}\right)  } \label{eq001}%
\end{equation}
We need to study the \emph{convergence} of (\ref{eq001}). E.g., for
$F:\mathbb{R}\rightarrow\mathbb{R}^{+}$, increacing function of $s$, is it a
contraction? Look at fixed point theorems (Nash?), derivatives etc. If we
cannot get theoretical results, maybe study numerically.

\item Can we replace \texttt{randsample} with a deterministic operation? Then
we will have a deterministioc system, perhaps easier to study.

\item Further (and more general, not just for duel tournaments)\ of the
imitation based dynamic.

\begin{enumerate}
\item This appears to be a very simple rule and it also looks like its
analysis will not be too complicated.

\item Also consider the case when at most $K\in\left\{  1,...,N\right\}  $
changes can take place per generation (or iteration).

\item Note:\ in this case, instead of evolutionary tournament and generations,
we can think of a \textquotedblleft process\textquotedblright\ and its iterations.

\item Variation of\ the imitation protocol: $P_{n}$ imitates the strategy of a
randomly chosen player (neighbor)\ $P_{m}$ with probability $\pi$ which
depends on the extended fitness values of involved partners:%
\[
\pi=\frac{1}{1+e^{-\frac{\text{Fit}\left(  n\right)  -\text{Fit}\left(
m\right)  }{K}}}%
\]

\end{enumerate}

\item Lainiotis strategy propagation:%
\[
z_{m,t+1}=\frac{z_{m,t}F\left(  \sum_{k\neq m}z_{k,t}Q_{mk}\right)  }%
{\sum_{m=1}^{M}z_{m,t}F\left(  \sum_{k\neq m}z_{k,t}Q_{mk}\right)  }%
\]

\end{enumerate}

\item \textbf{More generally}: what can evolutionary tournament (or
evolutionary Game Theory) tell us about NE\ selection?
\end{enumerate}

asas
\end{document}